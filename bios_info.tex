De BIOS of EUFI is verantwoordelijk voor het opstarten van de computer, het moet dus heel veel weten van de hardware in een systeem. Als we de BIOS zouden kunnen uitvragen over ons systeem dan zou dat heel veel informatie kunnen opleveren, potentieel zelfs heel gevoelige informatie. Je kan dan dit dan ook niet als gewone gebruiker. Je moet root zijn om \texttt{dmidecode}\index{demidecode}\index{commando!demidecode} te kunnen gebruiken. In de gegeven voorbeelden zullen we dna ook \texttt{sudo} gebruiken.

De totale informatie die aanwezig is in de DMI-tabel is heel veel, dus het is handiger om er alleen de stukjes uit te halen die we nodig hebben. Dit doen we door aan \texttt{demidecode} de optie \texttt{-s} mee te geven met het onderdeel dat we zouden willen hebben. Als we bijvoorbeeld alleen de informatie over de BIOS versie willen hebben gebruiken we:
\begin{lstlisting}[language=bash]
$ sudo dmidecode -s bios-version
GNET88WW (2.36 )
\end{lstlisting}
Als je alleen de \texttt{-s} optie meegeeft en verder niets dan zie je welke keywords er bestaan voor \texttt{-s}.

Als we iets meer willen weten van het BIOS kunnen we ook meer informatie op vragen. Alle informatie over het systeem is opgedeeld in types. Met \texttt{-t} kan je alle informatie van een type opvragen, zonder een keyword krijg je een lijst te zien van beschikbare types. Alle BIOS informatie valt onder het keyword bios.
\begin{lstlisting}[language=bash]
$ sudo dmidecode -t bios
# dmidecode 3.2
Getting SMBIOS data from sysfs.
SMBIOS 2.7 present.

Handle 0x002A, DMI type 13, 22 bytes
BIOS Language Information
	Language Description Format: Abbreviated
	Installable Languages: 1
		en-US
	Currently Installed Language: en-US

Handle 0x0040, DMI type 0, 24 bytes
BIOS Information
	Vendor: LENOVO
	Version: GNET88WW (2.36 )
	Release Date: 05/30/2018
	Address: 0xE0000
	Runtime Size: 128 kB
	ROM Size: 12288 kB
	Characteristics:
		PCI is supported
		PNP is supported
		BIOS is upgradeable
		BIOS shadowing is allowed
		Boot from CD is supported
		Selectable boot is supported
		ACPI is supported
		USB legacy is supported
		BIOS boot specification is supported
		Targeted content distribution is supported
		UEFI is supported
	BIOS Revision: 2.36
	Firmware Revision: 1.14
\end{lstlisting}
Je ziet dat we zo al heel veel meer informatie naar boven halen.

\texttt{dmidecode} commando waarmee je veel informatie kan vinden. Lees de manual-pagina eens door en speel eens met de \texttt{-t} en \texttt{-s} opties om ermee vertrouwd te raken.


