Linux kent verscheidene tools om een harddisk te beheren, \'e\'en van de oudste en meest gebruikte is \texttt{fdisk}\index{fdisk}\index{commando!fdisk}. \texttt{fdisk} heeft geen grafische interface, maar wel een character gebaseerde interface. Naast \texttt{fdisk} zijn er ook \texttt{cfdisk} en \texttt{sfdisk}. De \texttt{cfdisk} tool heeft een iets vriendelijkere interface, maar er kan iets minder mee en \texttt{sfdisk} is vooral in scripts.

Met het commando
\begin{lstlisting}[language=bash]
$ sudo fdisk -l
\end{lstlisting}
kan je een overzicht krijgen van alle disks en partities in het systeem inclusief hun grote.
