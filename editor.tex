Op de commandline heb je geen menu's en vaak geen muis om door een applicatie te navigeren. Het maken van documenten is dan ook een stuk lastiger dan in een grafische interface. Toch zijn er oplossingen gevonden om op de commandline te werken met bestanden. Een tekstverwerker op de commandline heet een editor. Er zijn verschillende editors bedacht en in gebruik. Een van de oudste voor Unix geschreven editors is \textbf{vi}.

Een van de grote namen achter Unix is Ken Thompson. De eerste drie commando's die hij schreef voor het jonge Unix systeem waren \textbf{as} (assembler), \textbf{ed} (editor) en \textbf{sh} (shell). Dennis M. Ritchie bracht verbeteringen aan op ed en vanaf 1969 tot 1976 bleef dit de editor op een Unix systeem. In 1976 kwam Billy Joy en Chuck Haley met een nieuwe editor die ex werd genoemd. Voor ex schreef Billy Joy ook een soort interface om er makkelijker mee te kunnen werken en die wrapper om ex noemde hij vi (visual interface). Vanaf 1979 werd ex geintergreerd in vi en was er alleen nog vi. Later werd vi onderdeel van de Single Unix Specification en daarmee een editor die op bijna elk Unix systeem aanwezig is en dat is nogsteeds het geval. Op bijna alle beschikbare Unix systemen, van BSD tot Linux en Mac OS X is vi aanwezig. Dat is dan ook het voordeel van het aanleren van het werken met vi dat de kennis op verschillende platformen gebruikt kan worden.
