\index{Manual pages!man-pages}
De op het systeem aanwezig help van een Unix-achtig systeem is te vinden in de man-pages. De online Manual is verdeeld
in hoofdstukken.

\begin{enumerate}
\item Executable programs or shell commands
\item System calls (functions provided by the kernel)
\item Library calls (functions within program libraries)
\item Special files (usually found in /dev)
\item File formats and conventions eg /etc/passwd
\item Games
\item Miscellaneous (including macro packages and conventions), e.g. man(7), groff(7)
\item System administration commands (usually only for root)
\item Kernel routines [Non standard]
\end{enumerate}

Voor gebruikers zijn de belangrijkste hoofdstukken die over de commando's gaan, 1 en 8, en die over de configuratie
bestanden 5. Om een beetje vertrouwd te raken met de man-pages gaan we de manual page over man bekijken. Type:

\begin{lstlisting}[language=bash]
$ man man
\end{lstlisting}

Met de pijltjes toetsten omhoog en naar beneden kan je door de pagina scrollen en met q verlaat je de manual-pagina.
Scrollend door de pagina kom je ook de hierboven al genoemde hoofdstuk indeling tegen. Verder kom je kopjes tegen met
hoofdletter. Veel voorkomende koppen zijn:

\begin{verbatim}
NAME: naam van het commando met een korte uitleg

SYNOPSIS: syntax van het commando

DESCRIPTION: een beschrijving van het commando

OPTIONS: welke opties kent het commando

EXAMPLES: voorbeelden hoe het commando te gebruiken

AUTHORS: wie heeft de manual-page geschreven

SEE ALSO: doorverwijzingen naar andere documentatie
\end{verbatim}

Om snel te zoeken naar bepaalde stukken in een manual-page kan je de / gebruiken. Als je een man-page open hebt staan en
je typt

\begin{lstlisting}[language=bash]
/EXAMPLES
\end{lstlisting}

en deze kop bestaat in de pagina dan kom je meteen bij de examples terecht. Snel door de man-pages scrollen kan door
gebruik te maken van PgUp en PgDn toetsen.

