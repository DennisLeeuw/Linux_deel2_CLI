Informatie over de kernel die actief is kan verkregen worden met het \texttt{uname}\index{uname}\index{commando!uname} commando. \texttt{uname} heeft verschillende opties (zie de manual-page). De belangrijkste optie voor ons is de \texttt{-r} optie, want die verteld het release nummer van de kernel:
\begin{lstlisting}[language=bash]
$ uname -r
4.19.0-13-amd64
\end{lstlisting}
in de release is het deel voor het eerste streepje (-) het versie nummer van de kernel-broncode die gebruikt is. De rest is afkomstig van de bouwers van de distributie en die informatie kan verschillen per distributie bouwer.

Met de \texttt{-v} optie kan je ook de datum achterhalen waarop deze kernel gecompileerd is:
\begin{lstlisting}[language=bash]
$ uname -v
#1 SMP Debian 4.19.160-2 (2020-11-28)
\end{lstlisting}
in dit geval dus 28 november 2020.
