Met de pijltjes toetsen kun je in de edit mode bewegen door een bestand. De pijltjes links en rechts bewegen de cursor een character per keer naar links of rechts, de op en neer pijltjes bewegen de cursor een regel op en neer.

In de command mode kan je met \textbf{\^} bewegen naar het begin van een regel en met \textbf{\$} naar het einde van de regel. Een grotere sprong is met \textbf{gg} naar het begin van een document en met \textbf{G} naar het einde van het document.

Met \textbf{w} spring je een woord vooruit.

Je kan ook springen naar een bepaalde regel. Hiervoor gebruik je bijvoorbeeld :10. Dit commando sprint naar regel 10 van boven.

Met zoek opdrachten kan je ook door een document bewegen. Het \textbf{/} teken geeft het begin van een zoekopdracht aan. Dit werkt alleen in de command-modus. Met \textbf{/zoeken} zoek je naar het woord zoeken in de tekst. Met de letter \textbf{n} kan je naar het volgende zoekresultaat springen en met \textbf{N} naar het voorgaande.
