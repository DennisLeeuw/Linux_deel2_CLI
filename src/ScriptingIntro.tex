De meeste besturingssystemen kennen een vorm van scripting. Een scripting taal is een programmeertaal die niet gecompileerd hoeft te worden voordat hij uitgevoerd kan worden. De vertaling naar machinecode vindt plaats op het moment dat het script wordt uitgevoerd. Voor de uitvoer van het script is daarom altijd een interpreter (vertaler) nodig die het script vertaalt naar iets dat de computer snapt.

Er zijn verschillende soorten scripting talen met de daarbij behorende interpreters. Voor Linux systemen zijn de belangrijkste:
\begin{itemize}
\item Shell scripting (bash of sh)
\item Perl
\item PHP
\item Python
\item Java
\end{itemize}

Naast een command interpreter is de shell\index{shell} op elk Linux-systemen ook een interpreter van een scripting taal, de zogenaamde shell-scripts\index{shell-script}. Op Linux systemen is de standaard shell \texttt{bash}\index{bash}\index{commando!bash} en op Mac OS X aanwezig, dus als je programma's kan schrijven die de shell begrijpt kan je ze op veel systemen toepassen. In de meest simpele vorm is een shell-script een lijstje met commando's. Door het script op te starten worden de commando's \'e\'en voor \'e\'en opgestart, maar er is veel meer mogelijk. In dit hoofdstuk ga je leren hoe je shell-scripts schrijft.
