Met \texttt{free} hebben we zien wat het geheugen gebruik is en wat er nog vrij is. Het zou natuurlijk mooi zijn als we kunnen zien wat er geheugen gebruikt. Het geheugen wordt gebruikt door processen. Een proces is alles wat er in het geheugen zit en dat min of meer continue draait. Een overzicht van de aanwezige processen kan je krijgen met het \texttt{ps} commando. Het \texttt{ps} commando zonder opties laat alleen de processen zien van de gebruiker waaronder je bent ingelogd. Om alle processen te zien die op een systeem draaien is er de \texttt{-e} optie. Dit is vaak een heel lange lijst met processen en die zijn gesorteerd op proces id:
\begin{lstlisting}[language=bash]
$ ps -e | head -10
  PID TTY          TIME CMD
    1 ?        00:00:44 systemd
    2 ?        00:00:01 kthreadd
    3 ?        00:00:00 rcu_gp
    4 ?        00:00:00 rcu_par_gp
    6 ?        00:00:00 kworker/0:0H-kblockd
    8 ?        00:00:00 mm_percpu_wq
    9 ?        00:02:43 ksoftirqd/0
   10 ?        00:11:19 rcu_sched
   11 ?        00:00:00 rcu_bh
\end{lstlisting}

