We staan nog steeds in de / directory als we daar ls typen dan zien we allemaal verschillende directories op ons scherm
verschijnen.

Net als met de standaardisatie van Unix in een POSIX standaard werden er in het begin op Linux Distributies soms
bestanden in verschillende directories neergezet. Dat is voor programma's die op die systemen moeten draaien niet
handig. Als de ene distributie /var/db heeft voor het plaatsen van databases en de ander /var/databases dan schept dat
verwarring. De oplossing die hiervoor gekomen is is de Filesystem Hierarchy Standard. Deze is beschikbaar op
\url{https://refspecs.linuxfoundation.org/fhs.shtml}. Hier gaan we heel globaal in op een aantal belangrijke
directories, mocht je alle ins en outs willen weten dan raden we je aan om het document een keer te lezen.

De basis van het bestandssysteem wordt bepaald door de / directory, ook de root genoemd omdat de vertakkende directories
op een boom structuur lijkt en het Engelse root betekend stam.

Een ls van / laat ons een aantal verschillende directories zien. Laten we beginnen met /boot/. Deze directory bevat
bestanden die cruciaal zijn voor het opstarten maar die geen commando zijn. Hier vinden we de kernel en bestanden die
behoren bij de bootloader.

De /dev/ directory bevat de namen van beschikbare devices. Devices worden worden besproken in het hoofdstuk over
devices. Dus daar gaan we later nog op in.

De /etc/ directory bevat de configuratiebestanden van het systeem. Als je een instelling wil wijzigen is dit de plek om
te gaan zoeken.

/home/ bevat de directories waarin gebruikers hun bestanden kunnen zetten. Een uitzondering hierop is de directory
waarin de root gebruiker (de baas of administrator van het systeem), zijn bestanden kan opslaan. Die directory is
/root/.

/var/ is de directory voor de systeem opslag van variabele data zoals bijvoorbeeld de logbestanden die je dan ook kan
vinden in /var/log/.

/srv/ bevat de data van de diensten die door het systeem wordt aangeboden. Data van web- of
ftp-servers kan hier gevonden worden.

