Het is belangrijk dat processen informatie kunnen doorgeven aan beheerders van systemen. Een proces moet kunnen melden dat er bijvoorbeeld iets fout is gegaan. Met een grafische applicatie is dat geen probleem, die geeft dan een pop-up op het scherm. Een niet grafische applicatie kan een bericht op de commandline achterlaten, maar een proces dat in de achtergrond draait kan dat niet, die moet zijn meldingen ergens anders kwijt. Dit soort processen schrijven hun meldingen naar een log. Logging is de Engelse term voor vastleggen, opschrijven. Om die logging gestructureerd te doen, en niet elk proces zijn eigen keuze te laten maken, is er een directory genaamd \texttt{/var/log} waarin de log bestanden worden opgeslagen.

Elk proces mag zelf in deze directory schrijven of het mag gebruik maken een de logserver op een Linux systeem. Door gebruik te maken van de logserver worden logberichten op een gestandaardiseerde manier opgeslagen op het systeem.
