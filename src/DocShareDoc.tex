Een andere plek waar we op ons systeem informatie kunnen vinden is in de /usr/share/doc directory. Met het commando `ls'
kan je een listing van een directory opvragen. Type

\begin{lstlisting}[language=bash]
$ ls /usr/share/doc
\end{lstlisting}

en een enorme lijst van directories vliegt over je scherm. Dit is de documentatie van elk pakket dat op het systeem
ge\"installeerd is. Een linux systeem is opgebouwd uit allerlei software pakketten die vanaf source code door de
distributie maker gecompileerd zijn. De informatie die met de source code meekwam, zoals de licentie-bestanden,
eventuele FAQ-bestanden, READMEs etc. die vind je terug in de subdirectories van /usr/share/doc. Belangrijk zijn vaak
de voorbeeld configuratie-bestanden.
