De kracht van shell scripts is dat je de commando's die je op de command line gebruikt direct kunt toepassen in de software die je schrijft. Misschien kunnen we zelfs stellen dat een shell script een lijstje is van commando's die voor later gebruik in de juiste volgorde in het script zijn geplaatst. De vraag is natuurlijk wel, waarom zouden we dat willen?

De belangrijkste reden is dat als er herhaalde handelingen gedaan moeten worden dat de computer dat nauwkeuriger kan dan een mens. Als je eenmaal een script hebt geschreven dat werkt kan je het meerdere keren achter elkaar laten uitvoeren zonder dat er typefouten ontstaan, terwijl typefouten bij mens de meest voorkomende fout is bij het uitvoeren van een commando.

Als het complexe handelingen zijn is het helemaal handig om dit vast te leggen in een script zodat je zeker weet dat je geen stappen over zal slaan.

En als laatste zijn er soms taken die herhaald uitgevoerd moeten worden op vaste dagen of tijden. Denk hierbij aan bijvoorbeeld backups. Het Linux systeem heeft hiervoor een stukje software dat \texttt{cron}\index{cron}\index{commando!cron} heet en dat op gezette tijden een commando kan uitvoeren. Door verschillende commando's in een script te plaatsen en \texttt{cron} te vertellen het script uit te voeren kunnen er verschillende handelingen in een keer uitgevoerd worden.
