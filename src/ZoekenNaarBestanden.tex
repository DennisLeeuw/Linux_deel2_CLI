Met zo'n grote boom van directories is het van belang dat je op een simpele manier bestanden terug kan vinden. Om te
zoeken naar bestanden of directories is er het commando `find'. De syntax van `find' ziet er zo uit:

\begin{lstlisting}[language=bash]
$ find \ [-H] \ [-L] [-P] [-D debugopts] [-Olevel] [starting-point...] [expression]
\end{lstlisting}

Dit is geknipt en geplakt uit de man-page waar je uitgebreide documentatie kan vinden over find. Om een idee te krijgen
hoe find werkt type:

\begin{lstlisting}[language=bash]
$ find \~{} -name ``Muziek'' -print
\end{lstlisting}

Bij mij op het systeem was het antwoord
\begin{lstlisting}[language=bash]
/home/dennis/LinuxCursus/Muziek
\end{lstlisting}

Bij jou is dennis natuurlijk weer vervangen door je eigen gebruikersnaam. Print is de standaard functie van find, en
daarom vonden de programmeurs van find dat als je geen opdracht meegeeft dat find dan ook print doet. Dus korter kan
het zo
\begin{lstlisting}[language=bash]
$ find \~{} -name ``Muziek''
\end{lstlisting}

Die \~{} is makkelijk als je in je home-directory wil zoeken, maar wat als je in bijvoorbeeld /etc/ staat en in die
directory wil zoeken? Daar is ook over nagedacht. We hebben al `..' gezien als een aanduiding voor een lager gelegen
directory, maar zo is er ook de `.' als we `deze'-directory bedoel. En met deze bedoelen we de directory waarin we nu
staan. Dus we kunnen ook het volgende doen:

\begin{lstlisting}[language=bash]
$ find . -name ``Muziek''
\end{lstlisting}

En voor wie niets tegen typen heeft mag je natuurlijk ook de hele directory opgeven
\begin{lstlisting}[language=bash]
$ find /home/dennis/ -name ``Muziek''
\end{lstlisting}

Met -name geven we op dat we naar een bestandsnaam zoeken en een bestandsnaam kan ook een directory zijn. Als we
onderscheidt willen maken tussen bestanden en directories kan kunnen we een -type meegeven:
\begin{lstlisting}[language=bash]
$ find . -type d -name ``Muziek''
\end{lstlisting}

Zal dezelfde output geven want type d is een directory, maar als doen
\begin{lstlisting}[language=bash]
$ find . -type f -name ``Muziek''
\end{lstlisting}

dan vinden we niets, want het type f is een regulier bestand en Muziek is een directory.

Om dat te testen doen we het volgende
\begin{lstlisting}[language=bash]
$ touch \~{}/LinuxCursus/Documenten/leeg\_bestand.txt
$ find . -type f -name ``leeg\_bestand.txt''
\end{lstlisting}

Met touch kan je een nieuw leeg bestand aanmaken en dat hebben we gedaan en daarna hebben we find naar dat nieuwe
bestand laten zoeken. Nou lijkt dit een wat onzinnige actie omdat we al weten waar het bestand is, maar het geeft ons
een optie om een andere functie van find te demonstreren, namelijk het zoeken naar lege bestanden op het systeem:
\begin{lstlisting}[language=bash]
$ find . -size 0
\end{lstlisting}
de -size optie geldt alleen voor bestanden, de -empty optie laat naast lege bestanden ook lege directories zien
\begin{lstlisting}[language=bash]
$ find . -empty
\end{lstlisting}

Het find commando kent nog veel meer opties zoals zoeken naar de datum en tijd waarop een bestand is aangemaakt of een
moment later of eerder dan een bepaalde datum en tijd. De man-page van find documenteert al deze verschillende opties
en op Internet is heel veel uitleg te vinden hoe je find met al deze opties kan gebruiken. Een laatste functie van find
willen je nog meegeven. De kan behalve met -print find ook andere dingen laten doen dan de gevonden elementen printen,
je kan find ook vertellen om een actie uit te voeren, zoals het deleten van de gevonden elementen:

\begin{lstlisting}[language=bash]
$ find . -size 0 -delete
\end{lstlisting}
een ls van LinuxCursus/Documenten/ zal laten zien dat het bestand leeg\_bestand.txt niet meer bestaat. Pas wel op, dit
kan gevaarlijke situaties opleveren.

\begin{lstlisting}[language=bash]
$ find LinuxCursus/ -empty -delete
\end{lstlisting}

Dit commando zoekt in de LinuxCursus naar alle directories en bestanden die leeg zijn. Bestanden
hadden we niet meer, maar er stonden nog wel twee directories in (Documenten en Muziek), beide directories waren leeg
en werden door find dan ook keurig verwijderd van het systeem. Toen kwam find nog de directory LinuxCursus tegen, en ja
die was inmiddels ook leeg, dus heeft find die ook weggegooid!
