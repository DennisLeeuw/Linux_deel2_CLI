Om te achterhalen welke processor er in een machine zit is er het commando \texttt{lscpu}\index{lscpu}\index{commando!lscpu}. Het geeft snel veel informatie over de processor:
\begin{lstlisting}[language=bash]
$ lscpu
Architecture:        x86_64
CPU op-mode(s):      32-bit, 64-bit
Byte Order:          Little Endian
Address sizes:       39 bits physical, 48 bits virtual
CPU(s):              8
On-line CPU(s) list: 0-7
Thread(s) per core:  2
Core(s) per socket:  4
Socket(s):           1
NUMA node(s):        1
Vendor ID:           GenuineIntel
CPU family:          6
Model:               60
Model name:          Intel(R) Core(TM) i7-4810MQ CPU @ 2.80GHz
Stepping:            3
CPU MHz:             1291.340
CPU max MHz:         3800.0000
CPU min MHz:         800.0000
BogoMIPS:            5586.67
Virtualization:      VT-x
L1d cache:           32K
L1i cache:           32K
L2 cache:            256K
L3 cache:            6144K
NUMA node0 CPU(s):   0-7
Flags:               fpu vme de pse tsc msr pae mce cx8 apic sep mtrr pge mca cmov pat pse36 clflush dts acpi mmx fxsr sse sse2 ss ht tm pbe syscall nx pdpe1gb rdtscp lm constant_tsc arch_perfmon pebs bts rep_good nopl xtopology nonstop_tsc cpuid aperfmperf pni pclmulqdq dtes64 monitor ds_cpl vmx smx est tm2 ssse3 sdbg fma cx16 xtpr pdcm pcid sse4_1 sse4_2 x2apic movbe popcnt tsc_deadline_timer aes xsave avx f16c rdrand lahf_lm abm cpuid_fault epb invpcid_single pti ssbd ibrs ibpb stibp tpr_shadow vnmi flexpriority ept vpid ept_ad fsgsbase tsc_adjust bmi1 avx2 smep bmi2 erms invpcid xsaveopt dtherm ida arat pln pts flush_l1d
\end{lstlisting}
Informatie per aanwezige CPU kan verkregen worden via \texttt{/proc/cpuinfo}\index{/proc/cpuinfo} maar dit levert over het algemeen niet meer informatie op dan wat \texttt{lscpu} al gegeven heeft. Met \texttt{cat} kan je de inhoud van \texttt{/proc/cpuinfo} weergeven.
