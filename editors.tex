De eerste redelijk gebruiksvriendelijke editor op Unix was vi. De vi editor kent twee modi. De eerste modus is de edit mode en de tweede is de command mode. Standaard start vi op in de command mode waarin je commando's kunt geven om bestanden te laden of op te slaan en waarin je functies als knippen en plakken kan uitvoeren. De edit modus is die waarin je je tekst invoert. Dit onderscheid maakt voor beginnende gebruikers vi soms verwarrend.

Naast vi zijn er ook andere editors voor Unix-achtige systemen ontwikkeld. De meeste bekende zijn pico en nano. Pico was de oorspronkelijke editor. Nano is ontwikkled door het GNU-project en is een vervanging van pico omdat pico een licentie had die "problematisch" was. Dat probleem is inmiddels opgelost, maar nano biedt zoveel extra mogelijkheden dat velen de voorkeur geven aan nano.

Het grote voordeel van nano ten opzichte van vi is zijn gebruiksvriendelijke interface. Nano kent geen edit en command mode zoals vi. Nano gebruikt control codes om commando's te geven en is direct beschikbaar voor de invoer van tekst van de gebruiker.
