Van oudsher werden Unix systemen veel gebruikt door programmeurs er zijn dan ook veel talen ontwikkeld en overgezet naar
Linux. Natuur is er een C-compiler. De meest gebruikte is die uit het GNU project die de GNU Compiler Collection (GCC)
heet omdat hij naar \index{C}C ook compilers bevat voor \index{C++}C++, \index{Objective-C}Objective-C,
\index{Fortran}Fortran, \index{Ada}Ada, \index{Go}Go en \index{D}D.

Ook voor scripting talen zijn er veel \foreignlanguage{english}{interpreters} aanwezig zoals voor PHP, Perl, Python en Java.

Daarnaast is er via verschillende kanalen nog veel meer te installeren.

