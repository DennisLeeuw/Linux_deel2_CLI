\documentclass[a4paper,12pt,twoside,openright,titlepage]{book}

%Additional packages
\usepackage[ascii]{inputenc}
\usepackage[T1]{fontenc}
\usepackage[dutch,english]{babel}
% Turn on indexing
\usepackage{imakeidx}
\usepackage{syntonly}
\usepackage[official]{eurosym}
%\usepackage[graphicx]
\usepackage{graphicx}
\graphicspath{ {./images/} }
\usepackage{float}
\usepackage{hyperref}
\hypersetup{colorlinks=true, linkcolor=blue, citecolor=blue, filecolor=blue, urlcolor=blue, pdftitle=, pdfauthor=, pdfsubject=, pdfkeywords=}
\usepackage{tabularx}
\usepackage{scrextend}
\addtokomafont{labelinglabel}{\sffamily}
\usepackage{listings}
\usepackage{adjustbox}
\usepackage{color}

\makeindex[columns=3,title=Alfabetische index,intoc]

% Define colors
\definecolor{ashgrey}{rgb}{0.7, 0.75, 0.71}

% Listing style
\lstset{
  backgroundcolor=\color{ashgrey},   % choose the background color; you must add \usepackage{color} or \usepackage{xcolor}; should come as last argument
  basicstyle=\footnotesize,        % the size of the fonts that are used for the code
  breakatwhitespace=false,         % sets if automatic breaks should only happen at whitespace
  breaklines=true,                 % sets automatic line breaking
  extendedchars=true,              % lets you use non-ASCII characters; for 8-bits encodings only, does not work with UTF-8
  frame=single,	                   % adds a frame around the code
  keepspaces=true,                 % keeps spaces in text, useful for keeping indentation of code (possibly needs columns=flexible)
  rulecolor=\color{black},         % if not set, the frame-color may be changed on line-breaks within not-black text
  showstringspaces=false,
  showspaces=false,                % show spaces everywhere adding particular underscores; it overrides 'showstringspaces'
}

% Uncomment for production
% \syntaxonly

% Style
\pagestyle{headings}

% Define document
\author{D. Leeuw}
\title{Linux introductie voor systeembeheerders - voor MBO niveau 4 en het LPI Linux Essentials examen: De CLI}
%\subtitle{Linux voor MBO niveau 4 en het LPI Linux Essentials examen}
%\subject{Een Praktische Gids}
\date{\today\\v.0.2.0}

\begin{document}

\selectlanguage{dutch}

\maketitle

\copyright\ 2020 Dennis Leeuw\\

\begin{figure}
	\includegraphics[width=0.3\textwidth]{CC-BY-SA-NC.png}
\end{figure}

\bigskip

\input{licentie}

%%%%%%%%%%%%%%%%%%%
%%% Introductie %%%
%%%%%%%%%%%%%%%%%%%

\frontmatter
\chapter{Over dit Document}
\input{OverDitDocument}

%%%%%%%%%%%%%%%%%
%%% De inhoud %%%
%%%%%%%%%%%%%%%%%
\tableofcontents

\mainmatter
\chapter{Inleiding}
Deze Linux cursus beoogt aan te sluiten bij het Linux Essentials examen van de LPI (Linux Professional Institute). De complete opleiding bestaat uit drie delen in het eerste deel installeren we CentOS als werkstation en leren we het Linux systeem kennen vanuit de grafische interface. In het tweede deel installeren we CentOS en zullen we meer leren over het gebruik van Linux op de command line en in het derde deel installeren we Debian en gaan we Linux inzetten als (web)server en zien we de interactie tussen Linux systemen.\par
Alle Linux systemen zullen ge\"installeerd worden als virtuele machine op Virtual Box (\url{https://www.virtualbox.org/}). De keuze voor Virtual Box is genomen omdat het gratis beschikbaar is en gebruikt kan worden op Windows, Mac OS X en Linux. Daarmee maakt niet meer uit welk besturingssysteem u nu gebruikt.

Voor de CentOS machine is 15G vrije schijfruimte nodig en voor het Debian systeem 8G daarmee is er een totaal aan 23 G vrije disk ruimte nodig. Voor elke machine hebben we 2G RAM nodig, dus een totaal van 4G moet vrij beschikbaar zijn.


\chapter{Shell scripting}\index{Shell scripting}
De meeste besturingssystemen kennen een vorm van scripting. Een scripting taal is een programmeertaal die niet gecompileerd hoeft te worden voordat hij uitgevoerd kan worden. De vertaling naar machinecode vindt plaats op het moment dat het script wordt uitgevoerd. Voor de uitvoer van het script is daarom altijd een interpreter (vertaler) nodig die het script vertaalt naar iets dat de computer snapt.

Er zijn verschillende soorten scripting talen met de daarbij behorende interpreters. Voor Linux systemen zijn de belangrijkste:
\begin{itemize}
\item Shell scripting (bash of sh)
\item Perl
\item PHP
\item Python
\item Java
\end{itemize}

Naast een command interpreter is de shell\index{shell} op elk Linux-systemen ook een interpreter van een scripting taal, de zogenaamde shell-scripts\index{shell-script}. Op Linux systemen is de standaard shell \texttt{bash}\index{bash}\index{commando!bash} en op Mac OS X aanwezig, dus als je programma's kan schrijven die de shell begrijpt kan je ze op veel systemen toepassen. In de meest simpele vorm is een shell-script een lijstje met commando's. Door het script op te starten worden de commando's \'e\'en voor \'e\'en opgestart, maar er is veel meer mogelijk. In dit hoofdstuk ga je leren hoe je shell-scripts schrijft.

\section{Waarom scripting?}
De kracht van shell scripts is dat je de commando's die je op de command line gebruikt direct kunt toepassen in de software die je schrijft. Misschien kunnen we zelfs stellen dat een shell script een lijstje is van commando's die voor later gebruik in de juiste volgorde in het script zijn geplaatst. De vraag is natuurlijk wel, waarom zouden we dat willen?

De belangrijkste reden is dat als er herhaalde handelingen gedaan moeten worden dat de computer dat nauwkeuriger kan dan een mens. Als je eenmaal een script hebt geschreven dat werkt kan je het meerdere keren achter elkaar laten uitvoeren zonder dat er typefouten ontstaan, terwijl typefouten bij mens de meest voorkomende fout is bij het uitvoeren van een commando.

Als het complexe handelingen zijn is het helemaal handig om dit vast te leggen in een script zodat je zeker weet dat je geen stappen over zal slaan.

En als laatste zijn er soms taken die herhaald uitgevoerd moeten worden op vaste dagen of tijden. Denk hierbij aan bijvoorbeeld backups. Het Linux systeem heeft hiervoor een stukje software dat \texttt{cron}\index{cron}\index{commando!cron} heet en dat op gezette tijden een commando kan uitvoeren. Door verschillende commando's in een script te plaatsen en \texttt{cron} te vertellen het script uit te voeren kunnen er verschillende handelingen in een keer uitgevoerd worden.

\section{Hello World - een eerste script}
We gaan een bestand maken met de naam texttt{hello\_world.sh}\index{hello\_world.sh}. Het is een goeie gewoonte om een bestandsnaam te eindigen met een extensie die weergeeft wat voor type bestand het is. Dit is voor Linux niet noodzakelijk, maar voor ons gebruikers is het handig om te weten dat we te maken hebben met een shell-script en dus eindigen we de bestandsnaam met .sh.

Om structuur in ons werk te houden maken we een directory scripts aan in onze home directoy.
\begin{lstlisting}[language=bash]
$ cd ~
$ mkdir scripts
$ cd scripts
\end{lstlisting}
Maak met \texttt{vi} het bestand hello\_world.sh aan en plaats in het bestand de volgende tekst:
\begin{lstlisting}[language=bash]
echo "Hello world!"
exit 0
\end{lstlisting}
Sluit \texttt{vi} af. We hebben nu een bestand met 2 commando's. Regel 1 zegt beeld Hello World! op het scherm af en regel 2 zegt verlaat dit script. Linux leest echter geen extensies om te bepalen wat het moet doen, maar het leest de eerste twee characters van een bestand om te bepalen wat er met een bestand gedaan moet worden. Deze eerste twee characters van een bestand heten het magic number\index{magic number}.

Het \texttt{file}\index{file}\index{commando!file} commando kan gebruikt worden om het magic nummer van een bestand te lezen en weer te geven wat voor soort bestand het is.
\begin{lstlisting}[language=bash]
$ file hello_world.sh
hello_world.sh: ASCII text
\end{lstlisting}
We zien dat Linux nog denkt dat het om een text bestand gaat. Het ziet het niet als een shell script. Voeg nu boven de \texttt{echo} regel een regel toe zodat je script er zo uit komt te zien:
\begin{lstlisting}[language=bash]
#!/bin/bash
echo "Hello world!"
exit 0
\end{lstlisting}
Als je de editor verlaten hebt en weer een \texttt{file} doet van hello\_world.sh dan zal je zien dat Linux nu weet dat het om een shell-script gaat.

Voor een script dat ge\"interpreteerd moet worden door een interpreter, dat kan een shell zijn, zijn de 2-character codes de she-bang\index{she-bang}: \#! of hexadecimaal 0x23 0x21. Achter de she-bang komt de interpreter die het script moet interpreteren. Als de shell deze character reeks tegen komt zal het de rest van de regel lezen om te bepalen aan welke interpreter (shell) het het script moet voeren. In ons voorbeeld wordt het script gegeven aan het programma /bin/bash en deze zal de twee opgegeven commando's uitvoeren.

Eerst wordt door echo \texttt{Hello world!} op het scherm gezet en daarna sluit het script af. Het \texttt{exit}\index{exit}\index{commando!exit}\index{Shell scripting!exit} commando is niet noodzakelijk, als de shell geen commando's meer tegen komt zal het automatisch afsluiten. Het voordeel van \texttt{exit} is dat je er een exit code aan kan meegeven zodat je kan laten weten dat het script goed (0) of fout (1) ge\"eindigd is. Doe je dit niet dan is de exit code de error-code van het laatst uitgevoerde commando.

Voer het script uit en geef direct daarna het commando
\begin{lstlisting}[language=bash]
$ echo $?
\end{lstlisting}
als je nu de regel met daarop \texttt{exit 0} vervangt door \texttt{exit 1} en dan weer het script draait en het commando
\begin{lstlisting}[language=bash]
$ echo $?
\end{lstlisting}
geeft dan zie welke invloed het exit commando heeft.

\section{Het starten van scripts}
Het script kan op twee verschillende manieren opgestart worden. Allereerst kunnen we script rechtstreeks aan de shell geven:
\begin{lstlisting}[language=bash]
$ bash ./hello_world.sh
\end{lstlisting}
De tweede mogelijkheid is dat we het script executable maken en daarna direct uitvoeren. We kunnen een script executable maken voor de eigenaar door het x-bit te zetten voor de user:
\begin{lstlisting}[language=bash]
$ chmod u+x ./hello_world.sh
\end{lstlisting}
Daarna kan de eigenaar het script direct uitvoeren
\begin{lstlisting}[language=bash]
$ ./hello_world.sh
\end{lstlisting}
Bij beide opties is het van belang om het complete pad mee te geven aan het script. In de voorbeelden konden we volstaan met ./ omdat het script in de directory staat waar we al in staan. Maar we zouden ook het volledige pad kunnen opgeven:
\begin{lstlisting}[language=bash]
$ /home/dennis/scripts/hello_world.sh
\end{lstlisting}
We kunnen ook het pad /home/dennis/scripts toevoegen aan onze PATH variabele en er zo voor zorgen dat we nooit meer het pad hoeven op te geven.
\begin{lstlisting}[language=bash]
$ export PATH=$PATH:/home/dennis/scripts
$ hello_world.sh
\end{lstlisting}

\section{Commentaar}\index{Shell scripting!commentaar}
Op het moment dat je een programma schrijft is het meestal volledig duidelijk wat je aan het doen bent en waarom. Als het script complexer wordt is het soms al lastiger en als je er na een jaar naar kijkt heb je soms geen idee meer waarom je het hebt gemaakt en hoe het ook alweer in elkaar zat. Het is daarom goed om uitleg in je script te verwerken. Dit maakt het voor jezelf duidelijker, maar ook als iemand anders iets wil wijzigen aan wat jij gemaakt hebt maakt het het voor die persoon een stuk makkelijker als hij of zij weet wat er waar gebeurd in het script.
Open met vi het hello\_world.sh\index{hello\_world.sh} script en wijzig het zo dat het er zo uit komt te zien:
\begin{lstlisting}[language=bash]
#!/bin/bash
# (c) 2020, Dennis Leeuw
# License: GPL v3
# Versie: 1.0
# Dit script print Hello world! Op het scherm

echo "Hello World!"
# Verlaat het script met error-code 0
exit 0
#END
\end{lstlisting}
Je hebt nu 5 regels commentaar toegevoegd. Het \# symbool geeft dat het commentaar is en dat de interpreter (de shell) er niets mee moet doen. Als je het script opnieuw uitvoert zie je dat er niets gewijzigd is aan de uitvoer.

Vervang aan het commentaar het jaartal 2020 naar het jaartal waarin je nu leeft en verander de naam Dennis Leeuw in je eigen naam.

Het is een goede gewoonte om aan te geven wie een script gemaakt heeft, wat de licentie is en een korte beschrijving te geven van wat het script doet. Je werkt waarschijnlijk niet je leven lang op dezelfde afdeling of waarschijnlijk niet eens bij hetzelfde bedrijf. Beheerders na jou kunnen op deze manier snel zien wie iets gemaakt heeft en wat het script doet.

Ik laat mijn scripts meestal eindigen met \#END zodat ik weet of het script compleet is. Als het script begint met een she-bang en eindigt met \#END dan is het script compleet.

\section{Variabelen}\index{Shell scripting!variabelen}
Het is soms handig om variabelen gebruiken:

\begin{lstlisting}[language=bash]
$ aap=1
$ echo $aap
\end{lstlisting}

Zoals je ziet gebruiken we bij het toewijzen van een waarde aan een variable (stop 1 in de variabele aap) geen \$-teken voor de variabele. Alleen bij het gebruik van de variabele zetten we er een \$-teken voor.

Variabelen in de shell hebben ook geen type, er bestaat geen integer of een string, dus we kunnen net zo makkelijk
doen:

\begin{lstlisting}[language=bash]
$ aap=''Dag aap!''
$ echo $aap
\end{lstlisting}

Tot nog toe heb je kennis gemaakt met twee variabelen van de shell. PATH en ?. Er zijn er nog veel meer die standaard beschikbaar zijn. Om er een paar te noemen:
\begin{lstlisting}[language=bash]
$ echo $USER
$ echo $SHELL
$ echo $HOME
$ echo $PWD
\end{lstlisting}
om een complete lijst te krijgen van alle variabelen die in je huidige sessie tot je beschikking staan is er het commando \texttt{env}\index{env}\index{commando!env}.
Als je de waarde van een variabele wil wijzigen gebruik je \texttt{export}\index{export}\index{commando!export}:
\begin{lstlisting}[language=bash]
$ echo $PATH
$ PATH=".:${PATH}"
$ export PATH
$ echo $PATH
\end{lstlisting}
met deze opdracht hebben we de . directory toegevoegd aan de PATH variabele. Als we een commando aanroepen en het komt voor in de directory waar we op dat moment in staan dan zal het dat commando uitvoeren. We hoeven dan niet meer het hele pad of de ./ op te geven.

Een variabele is een plek om data in op te slaan voor later gebruik. We kunnen ons script aanpassen op de volgende manier om een voorbeeld te geven van het gebruik van een variabele in scripts:
\begin{lstlisting}[language=bash]
#!/bin/bash
# (c) 2020, Dennis Leeuw
# License: GPL v3
# Versie: 1.1
# Dit script print Hello world! Op het scherm

output="Hello world!"
echo $output
# Verlaat het script met error-code 0
exit 0
#END
\end{lstlisting}

We hebben nu een variabele met de naam output toegevoegd en deze variabele hebben we de waarde (data) "Hello world!" gegeven.

Bij het echo-commando\index{echo} gebruiken we de variabele om weer onze vertrouwde uitvoer te krijgen.

Let er op dat bij het declareren (het van een waarde voorzien) van de variabele er geen dollar-teken voor de variabele staat, maar bij het gebruik wel.

De shell heeft ook een aantal ingebouwde variabelen\index{ingebouwde variabelen}\index{builtin variables}\index{variabelen!ingebouwde}. Deze variabelen worden met allemaal hoofdletters geschreven. Om verwarring te voorkomen is het dus handig om je eigen variabelen niet met allemaal hoofdletters te schrijven. De meest gebruikte shell-variabelen in scripts zijn \$USER die de gebuikersnaam bevat, \$HOME die de home-directory bevat en \$PWD die het complete directory pad bevat vanaf de root tot de directory waarin de gebruiker stond toen hij het script startte. Probeer dit script uit:
\begin{lstlisting}[language=bash]
#!/bin/bash
# (c) 2020, Dennis Leeuw
# License: GPL v3
# Versie: 1.1
# Vertelt de gebruiker met welke gebruikersnaam hij of zij ingelogd is, wat
# zijn of haar home-directory is en in welke directory hij of zij nu staat.

echo "Je bent ingelogd als $USER en je home-directory is $HOME"
echo "Je huidige directory is $PWD"

exit 0
#END
\end{lstlisting}

Variabelen worden dus gebruikt om data in op te slaan. Je kan ook de uitvoer van een commando in een variabele stoppen. Op Linux bestaat er het commando \texttt{date}\index{date}\index{commando!date} dat op verschillende manieren informatie over de datum en tijd kan weergeven en waarmee je de datum en tijd op je computer kan zetten. Omdat date zoveel kan lijkt het in eerste instantie ingewikkeld, maar als je er een beetje ervaring mee op doet blijkt het een enorm handige tool te zijn. Tikken we alleen date in op de commandline en daarna enter, dan krijgen we van het systeem de huidge datum en tijd terug en in welke tijdzone we leven. Als we bijvoorbeeld alleen het jaar willen weten waarin we nu leven dan vragen we dat aan date door:
\begin{lstlisting}[language=bash]
date +%Y
\end{lstlisting}
en alleen de maand is:
\begin{lstlisting}[language=bash]
date +%m
\end{lstlisting}
dit is het maandnummer, willen de naam van de maand dan gebruiken we:
\begin{lstlisting}[language=bash]
date +%B
\end{lstlisting}
Meer van deze opties vind je terug in de manual-pagina van texttt{date}. Lees deze door zodat je ook weet hoe je de dag, de uren en de minuten moet weergeven.

Om in een script de uitvoer van \texttt{date} in een variabele te zetten kunnen we het volgende gebruiken:
\begin{lstlisting}[language=bash]
#!/bin/bash
# (c) 2021, Dennis Leeuw
# License: GPL v3
# Versie: 1.0
# Wat is de huidige datum en tijd

start="$(date '+%Y-%m-%d-%H-%M')"

echo ${start}

exit 0
#END
\end{lstlisting}
We hebben nu de uitvoer van het \texttt{date} commando aan een variabele gegeven en deze geprint. Let op de quotes! Als we parameters aan een commando willen voeren die ook een betekenis in de shell hebben, zoals in dit geval het \%-teken, dan moeten we de parameter tussen enkele quotes zetten zodat ze door de shell niet ge\"interpreteerd worden.

De laatste manier om een variabele een waarde te geven is door aan de gebruiker input te vragen, hiervoor bestaat er het \texttt{read}\index{read}\index{commando!read} commando. In een script zou je dat zo kunnen gebruiken:
\begin{lstlisting}[language=bash]
#!/bin/bash
# (c) 2021, Dennis Leeuw
# License: GPL v3
# Versie: 1.0
# Vraag de gebruiker om zijn leeftijd

echo -n "Hallo $USER, hoe oud ben jij? "
read leeftijd

echo "De gebruiker $USER is $leeftijd jaar oud."

exit 0
#END
\end{lstlisting}


\subsection{Opdracht: Werken met variabelen}
\begin{enumerate}
\item Maak een script met twee variabelen. Je voornaam stop je in de eerste variabele en je achternaam in de andere.
	
	Zorg ervoor dat het script op je scherm je voornaam en achternaam achter elkaar laat zien.
\item Schrijf een script dat aan een variabele het weeknummer geeft door gebruik te maken van het \texttt{date} commando. Normaal komt er een 2 cijferig getal uit, bij een cijfer onder de 10 wordt er een 0 voor het getal gezet. Schrijf het \texttt{date} commando zo dat er nooit een 0 voor gezet wordt.
\item Schrijf een script waarbij aan de variabele V1 de waarde 'Dit is een zin' wordt toegekend en maak een tweede variabele V2 met de waarde 'en die eindigt zo.". Een derde variabele V3 moet nu de waarde van V1 en V2 bevatten en op het scherm worden afgebeeld.
\end{enumerate}

\section{if}\index{if}\index{Shell scripting!if}
Je kan een script gebruiken om alle commando's die je op de commandline gebruikt onder elkaar te zetten als een lange lijst. Beter is het als je ook controleert of een commando succesvol is verlopen. Eerder hebben we behandeld dat de exit code van een commando terecht komt in de variabele \$?. Als we een test zouden kunnen doen in een script om te zien of deze variabele 0 is, dus als alles goed is verlopen, dan kunnen we controleren of er commando's fout zijn gegaan.

Er is een commando dat \texttt{test}\index{test}\index{commando!test} heet dat doet wat het zegt, je kan er dingen mee testen, en de uitkomst is waar (true) of nietwaar (false), andere opties zijn er niet. Dit commando gaan we niet direct in een script gebruiken, maar eerst eens op de commandline uitproberen om te zien wat we ermee kunnen.
\begin{lstlisting}[language=bash]
$ touch test_bestand.txt
$ test -f test_bestand.txt
$ echo $?
$ test -f deze_bestaat_niet
$ echo $?
$ file -s test_bestand.txt
$ echo $?
\end{lstlisting}
Lees de manual-page van \texttt{test} eens door en zoek uit wat de \texttt{-s} optie betekent.

Het \texttt{test} commando kan ook gebruikt worden om waarden met elkaar te vergelijken. \texttt{test} maakt daarin in onderscheid tussen strings en integers (getallen).
\begin{lstlisting}[language=bash]
$ test "Aap" = "Aap"; echo $?
$ test "Aap" = "Aapje"; echo $?
$ test "01" = "1"; echo $?
$ test "01" -eq "1"; echo $?
\end{lstlisting}
Zoals je gezien zult hebben test de = op het gelijk zijn van strings en -eq op het gelijk zijn van integers, waarbij 01 gelijk is aan 1.

Strings kan je vergelijken met = om te zien of ze aan elkaar gelijk zijn en met != om te zien of ze niet aan elkaar gelijk zijn.

Met integers kan je nog veel meer vergelijkingen maken:
\begin{description}
	\item[-eq] Controlleer of integers gelijk zijn (equal)
	\item[-ne] Controlleer of integers ongelijk zijn (not equal)
	\item[-gt] Controlleer of eerste integer groter is dan de tweede (greater then)
	\item[-ge] Controlleer of de eerste integer groter of gelijk is aan de tweede (greater or equal)
	\item[-lt] Controlleer of eerste integer kleiner is dan de tweede (less then)
	\item[-le] Controlleer of de eerste integer kleiner of gelijk is aan de tweede (less or equal)
\end{description}
In de manual pagina vind je nog veel meer mogelijkheden waarop je kan testen met \texttt{test}, maar voor ons zijn dit de belangrijkste opties. We gaan de opgedane kennis toepassen in shell-scripts.

Het eerste script dat we gaan maken is een script dat een \texttt{ping}\index{Shell scripting!ping} doet naar een IP-adres en dat test of de ping goed verlopen is:
\begin{lstlisting}[language=bash]
#/bin/bash
# (c) 2021, Dennis Leeuw
# ping een host en kijk of dat succesvol is verlopen

ip="192.168.10.42"

ping -c1 ${ip} 2>/dev/null 1>/dev/null
if [ $? -ne 0 ]
	then
	echo "Het ${ip} is niet aanwezig op het netwerk"
fi

exit 0
#END
\end{lstlisting}
We voeren hier het \texttt{ping}\index{ping}\index{commando!ping} commando uit en sturen de error en andere berichten naar \texttt{/dev/null}. Als de return code in \$? niet 0 is, dus 1 of hoger is, dan heeft onze ping geen antwoord gekregen en vertellen we dat aan de gebruiker.

Het zal opgevallen zijn dat we bij de \texttt{if} twee bijzondere tekens hebben neergezet [ en ]. Deze tekens betekenen dat er een test gedaan moet worden. We hadden de regel ook zo kunnen schrijven:
\begin{lstlisting}[language=bash]
if test $? -ne 0
\end{lstlisting}
Het is echter een goede gewoonte om in scripts de blokhaken, [ en ], te gebruiken.

Stel dat we ook aan de gebruiker willen laten weten als het wel goed gegaan is. Dus in alle andere gevallen, dat kunnen we doen met \texttt{else}\index{else}\index{Shell scripting!else}:
\begin{lstlisting}[language=bash]
#/bin/bash
# (c) 2021, Dennis Leeuw
# ping een host en kijk of dat succesvol is verlopen

ip="192.168.10.42"

ping -c1 ${ip} 2>/dev/null 1>/dev/null
if [ $? -ne 0 ]
	then
	echo "Het ${ip} is niet aanwezig op het netwerk"
	else
	echo "Het ${ip} is wel op het netwerk aangetroffen"
fi

exit 0
#END
\end{lstlisting}
Probeer dit script ook eens met een ander IP adres, bijvoorbeeld 127.0.0.1.

De \texttt{if, then, else} kent in bash nog een toevoeging namelijk de \texttt{elif}. De \texttt{elif} is een samentrekking van \texttt{else} en \texttt{if} en kan gebruikt worden om meerdere tests na elkaar te doen. Bijvoorbeeld:
\begin{lstlisting}[language=bash]
#/bin/bash
# (c) 2021, Dennis Leeuw
# test een getal

getal=15

if [ ${getal} -le 10 ]; then
	echo "Getal is kleiner of gelijk aan 10"
elif [ ${getal} -ge 20 ]; then
	echo "Getal is groter of gelijk aan 20"
else
	echo "Getal ligt tussen 10 en 20"
fi
\end{lstlisting}
Neem dit script over en varieer de waarde van de variabele \texttt{getal}, zodat het het script een keer door de \texttt{if} loopt, een keer door de \texttt{elif} en een keer door de \texttt{else}.

\subsection{Opdracht: Werken met if}
\begin{enumerate}
\item Maak een script dat aan de gebruiker een getal vraagt en geef weer of het getal kleiner of gelijk is aan 10, groter is dan 10 en kleiner of gelijk is aan 20, groter is dan 20 en kleiner of gelijk is aan 30 of groter is dan 30.
	
\item Maak een script dat test of een bestand leeg is. Als het bestand leeg is moet het verwijderd worden.

\item Maak een script dat aan de gebruiker om twee bestandsnamen vraagt. Test of de bestanden bestaan en kijk dan wel bestand nieuwer is. Laat aan de gebruiker weten welk bestand het nieuwste is. Tips: lees de manual-pagina van \texttt{test} en om te testen lees de manual-pagina van \texttt{touch}.
\end{enumerate}

\section{for}\index{for}\index{Shell scripting!for}
Om een handeling een bepaald aantal keren uit te voeren kunnen we gebruik maken van \texttt{for}. Met behulp van \texttt{for} kunnen we een loop doorlopen. Het \texttt{for} commando heeft twee hulpcommando's nodig, namelijk \texttt{do}\index{Shell scripting!do} en \texttt{done}\index{Shell scripting!done}. Na het \texttt{for} commando komt de conditie waaraan we moeten voldoen en tussen \texttt{do} en \texttt{done} staat de loop die doorlopen wordt. Laten we beginnen met een voorbeeld:
\begin{lstlisting}[language=bash]
#!/bin/bash
# (C) 2021 Dennis Leeuw
# License: GPLv3
# Tel van 1 tot 10

list="1 2 3 4 5 6 7 8 9 10"

for i in ${list}
do
	echo $i
done

#END
\end{lstlisting}
In dit script wordt op basis van een lijstje getallen elk getal \'e\'en voor \'e\'en aan de variabele i toegekend, dit is de conditie waaraan de \texttt{for} loop moet voldoen. Zolang er nog een parameter (getal) is wordt de loop tussen \texttt{do} en \texttt{done} doorlopen.

Tussen de commando's die het begin (\texttt{do}) en het einde van de loop (\texttt{done}) aangeven staan de commando's die in de loop uitgevoerd moeten worden. In ons geval is dat maar \'e\'en commando namelijke echo \$i.

Wat je misschien opgevallen is is dat de shell automatische weet dat hij de getallen 1 tot en met 10 als losse waarden moet beschouwen en deze aan de variabele i moet toekennen. We zouden het script kunnen herschrijven met quotes en zien wat er dan gebeurd:
\begin{lstlisting}[language=bash]
#!/bin/bash
# (C) 2021 Dennis Leeuw
# License: GPLv3
# Tel van 1 tot 10

list="'1 2' 3 4 '5 6' 7 8 '9 10'"

for i in ${list}
do
	echo $i
done

#END
\end{lstlisting}
Neem dit script over, let goed op de enkele quotes, en start het script. Ook hier geldt dus wat de enkele quotes (ticks) ervoor zorgen dat spaties niet meer gebruikt worden als scheidingsteken.

De vraag die hopelijk nu ontstaan is is waarom gebruikt de shell een spatie als scheidingsteken? Dat komt omdat de shell een interne variabele kent die \texttt{IFS}\index{IFS}\index{Internal Field Separator} heet en dat staat voor Internal Field Separator (interne veld scheider). Deze variabele bepaalt hoe de shell met velden omgaat. Standaard gebruikt de shell, spaties, tabs en de nieuwe regel (enter) als scheidingstekens. Ik kan het bovenstaande script dan ook herschrijven met nieuwe regels zonder dat de werking veranderd:
\begin{lstlisting}[language=bash]
#!/bin/bash
# (C) 2021 Dennis Leeuw
# License: GPLv3
# Tel van 1 tot 10

list="1
2
3 4
5
6 7
8
9 10"

for i in ${list}
do
	echo $i
done

#END
\end{lstlisting}
Het is minder goed leesbaar, maar voor de werking maakt het niet uit. We kunnen de IFS ook aanpassen. We maken nu een nieuw script waarin we het scheidingsteken (\texttt{IFS}) naar een newline (nieuwe regel) zetten en de \texttt{list} variabele ophakken in twee stukken:
\begin{lstlisting}[language=bash]
#!/bin/bash
# (C) 2021 Dennis Leeuw
# License: GPLv3
# Tel van 1 tot 10

# Set the IFS variable to newline
IFS=$'\n'

list="1 2 3 4 5
6 7 8 9 10"

# Count from 1 to 10 ... or not
for i in ${list}
do
	echo $i
done

#END
\end{lstlisting}
De uitvoer zal er dan zo uitzien
\begin{lstlisting}[language=bash]
$ ./script_for_IFS.sh 
1 2 3 4 5
6 7 8 9 10
\end{lstlisting}
Alleen de newline scheidt dan nog de delen en de spaties worden niet meer gezien als scheidingsteken.

Je kan de \texttt{IFS} naar elk willekeurig character zetten:
\begin{lstlisting}[language=bash]
#!/bin/bash

IFS="n"
zin="Dit is een zin en daarin komt vaak de n voor"
for i in $zin
do
	echo $i
done
\end{lstlisting}
De n is nu het scheidingsteken geworden en dat betekent dat de uitvoer van dit script er zo uit komt te zien:
\begin{lstlisting}[language=bash]
$ ./testscriptIFS.sh
Dit is ee
 zi
 e
 daari
 komt vaak de
 voor
\end{lstlisting}
Je ziet dat alle n-en zijn verdwenen en dat de zin is opgehakt op de plek waar de n stond.

\subsection{Opdracht: Werken met for}
Voor elke nieuwe gebruiker willen we vast een aantal directories maken binnen het bedrijf waarin we werken. Als de directory al aangemaakt is moet die niet gemaakt worden en als hij nog niet bestaat moet hij wel aangemaakt worden. De input van het script moet de naam van de gebruiker zijn. Maak gebruik van \texttt{for} om in een loop de directories aan te maken. De directory-tree die aangemaakt moet worden is:
\begin{itemize}
\item Documenten/Werkoverleg
\item Documenten/Werkoverleg/Verslagen
\item Documenten/Werkoverleg/Actiepunten
\item Documenten/Handleidingen
\item Documenten/Klanten
\item Documenten/Klanten/Systeem\_documentatie
\item Documenten/Servers/Systeem\_documentatie
\item Documenten/Servers/Inkoop\_documenten
\item Documenten/Servers/Handleidingen
\item Scripts
\end{itemize}

\section{while}\index{while}\index{Shell scripting!while}
Om door een lijst met elementen te lopen tot een bepaalde conditie is bereikt gebruiken we \texttt{while}. \texttt{while} kent hetzelfde begin en eind van de loop als \texttt{for}, namelijk \texttt{do} en \texttt{done}. Een voorbeeld van het gebruik van \texttt{while} zou kunnen zijn:
\begin{lstlisting}[language=bash]
#!/bin/bash
# (C) 2021 Dennis Leeuw
# While voorbeeld

i=1
while [ $i -le 10 ]
do
	echo $i
	i=$(($i+1))
done

#END
\end{lstlisting}
Dit script lijkt erg op de \texttt{for}-loop uit de vorige paragraaf. We vertellen hier dat de variabele \texttt{i} 1 is en vragen aan \texttt{while} om de loop te doorlopen zolang \texttt{i} kleiner of gelijk is aan 10. Om te zorgen dat \texttt{i} bij elke keer dat de loop doorlopen wordt 1 hoger wordt, moeten we bij \texttt{i} 1 optellen. Rekenen in de shell doen we door gebruikt te maken van dubbele haken, de uitkomst van de berekening zetten we weer in \texttt{i} waardoor \texttt{i} 1 hoger wordt. Deze manier van rekenen kan je ook op de commandline gebruiken, daar heb je niet alleen de shell voor nodig.

In de opgave uit de vorige paragraaf heb je een script gemaakt dat bepaalde directories aanmaakt op basis van een gebruikersnaam die wordt ingegeven door de gebruiker. Van de Linux commando's ben je inmiddels gewend dat je met opties waarden kan meegeven. Het zou natuurlijk heel mooi zijn als je die gebruikersnaam ook met een optie mee zou kunnen geven aan een script. En dat kan ook. In scripting heten parameters of opties die je meegeeft bij het opstarten van een script positional parameters\index{positional parameters}\index{Shell scripting!positional parameters}. Deze positional parameters hebben nummers en lopen van 0 tot het getal van de laatste parameter. Positional parameter 0 is altijd de naam van het script inclusief het pad waarmee het is opgestart.
\begin{lstlisting}[language=bash]
#!/bin/bash
# (C) 2021 Dennis Leeuw
# Positional parameters

echo "Dit script is opgestart als: $0"

echo "Parameter 1 is: $1"
echo "Parameter 2 is: $2"
echo "Parameter 3 is: $3"

#END
\end{lstlisting}
als we dit script opstarten met wat parameters kan krijgen we dit:
\begin{lstlisting}[language=bash]
$ ./scriptWhilePosPar.sh aap noot mies
Dit script is opgestart als: ./scriptWhilePosPar.sh
Parameter 1 is: aap
Parameter 2 is: noot
Parameter 3 is: mies
\end{lstlisting}
De eerste optie is in \$1 terecht gekomen, de tweede in \$2 enzovoort.

In dit script is het natuurlijk onhandig dat we voor elke positionele parameter een regel schrijven. Het zou makkelijker zijn als we de positionele paramaters \'e\'en voor \'e\'en zouden kunnen doorlopen. Een commando dat handig is bij het gebruik van positional parameters is \texttt{shift}\index{shift}\index{Shell scripting!shift}. Met \texttt{shift} verschuiven we de positional parameters een positie op. De waarde van parameter 3 wordt de waarde van parameter 2, de waarde van parameter 2 wordt de waarde van parameter 1 en de waarde van parameter 1 wordt weggegooid. Dit kunnen we gebruiken om zo door de hele lijst aan positional parameters te lopen.

\begin{lstlisting}[language=bash]
#!/bin/bash
# (C) 2021 Dennis Leeuw
# While voorbeeld 2

echo "Dit script is opgestart als: $0"

i=1
while [ "$1" != "" ]
do
	echo "Parameter $i is: $1"
	shift
	i=$(($i+1))
done

#END
\end{lstlisting}
De uitvoer van dit script is hetzelfde als het vorige, maar we hoeven nu niet meer voor elke parameter een regel te maken, plus dat het ook niet meer uitmaakt hoeveel parameters we meegeven. Met het gebruik van \texttt{shift} en \texttt{while} kunnen we de positionele parameters doorlopen opzoek naar opties die meegegeven zijn aan een script.

\subsection{Opdracht: Werken met While}
Maak een script dat werkt met de volgende 3 opties:
\begin{itemize}
	\item Als er een optie -u meegegeven wordt, dan is wat er achter -u komt een gebruikersnaam
	\item Als er een optie -d meegegeven wordt, dan is wat er achter -d komt een directory
	\item Als er een optie -h gegegeven wordt dan wordt er een beschrijving van het script weergegeven en stopt het script.
\end{itemize}
De opties -u en -d mogen in willekeurig volgorde aan het script meegegeven worden. Laat als resultaat op het scherm zien welke gebruikersnaam en welke directory als input gegeven zijn.


%%%%%%%%%%%%%%%%%%%%%
%%% Index and End %%%
%%%%%%%%%%%%%%%%%%%%%
%\backmatter
\printindex
\end{document}

%%% Last line %%%
