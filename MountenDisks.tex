Een disk of partitie op een disk wordt gemount op het bestandssysteem. Om dit duidelijk te maken moeten we wat dieper in
het systeem duiken en een paar dingen doen die we pas later in dit boek uitleggen. Dus type nauwkeurig het volgende
commando over:

\begin{lstlisting}[language=bash]
$ mount | grep ' / '
\end{lstlisting}

Voor en achter de / staat een enkele spatie. Dit geeft iets terug als:
\begin{lstlisting}[language=bash]
/dev/sda1 on / type ext4 (rw,relatime,errors=remount-ro)
\end{lstlisting}

Wat dit betekent is dat de eerste partitie op de harddisk die het systeem kent als sda gekoppeld (gemount) is op /.
USB-disks kunnen op dezelfde manier gekoppeld worden aan het systeem. De directory die daarvoor gereserveerd is is
/media/. In de /media/ directory zijn er directories per gebruiker waar gebruikers hun USB-disks kunnen mounten. Later
zullen we hier uitgebreider op terug komen.

/mnt is de directorie die van oudsher aanwezig is om lokaal media op te mounten en zou nu alleen nog door de beheerder
gebruikt moeten worden voor het tijdelijk mounten van disks.
